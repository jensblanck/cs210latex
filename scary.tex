\documentclass[notheorems]{beamer}%[handout]

\mode<presentation>
{
%  \useinnertheme[shadow]{rounded}
  \useinnertheme{rounded}
  \usecolortheme{rose,seahorse}
%  \usecolortheme{albatross}
  \setbeamercovered{transparent}
%  \usefonttheme[stillsansserifsmall,stillsansseriflarge]{serif}
  \usefonttheme{serif}
}

%\setbeameroption{show only notes}

%\usepackage[english]{babel}
\usepackage[utf8]{inputenc}
\usepackage{palatino}
\usepackage{tikz}
\usetikzlibrary{arrows,shapes}

\theoremstyle{definition}
\newtheorem{theorem}{Theorem}
\newtheorem{lemma}[theorem]{Lemma}
\newtheorem{proposition}[theorem]{Proposition}
\newtheorem{corollary}[theorem]{Corollary}
\newtheorem{definition}[theorem]{Definition}
\newtheorem{example}[theorem]{Example}
\newtheorem{remark}[theorem]{Remark}
\newtheorem{fact}[theorem]{Fact}
\newtheorem{question}[theorem]{Question}
%\theoremstyle{proof}
%\theoremstyle{example}
%\newtheorem{sketch}[theorem]{Sketch}

\DeclareMathOperator{\loglog}{loglog}

\newcommand{\nat}{\ensuremath{\mathbb{N}}}
\newcommand{\rat}{\ensuremath{\mathbb{Q}}}
\newcommand{\reals}{\ensuremath{\mathbb{R}}}
\newcommand{\ints}{\ensuremath{\mathbb{Z}}}
\newcommand{\complex}{\ensuremath{\mathbb{C}}}


\title%[Domain representations of topological algebras]
{Scary Lecture}

\author%[Author, Another]
{Jens Blanck}

\institute% (optional, but mostly needed)
{%
%\pgfuseimage{university-logo}\\
%Swansea University
   Swansea University
}

\date%[Short Occasion]
{}

\subject{Talks}

%\pgfdeclareimage[height=0.5cm]{university-logo}{../public_html/cslogo}
% \pgfdeclareimage[height=1cm]{wimcs-logo}{WIMCS_Logo}
% \logo{\pgfuseimage{wimcs-logo}}

%\pgfdeclareimage[height=4cm]{turing}{Turing}

\begin{document}

\begin{frame}
  \titlepage
  \includegraphics[height=3cm]{spok.jpg}
\end{frame}

\begin{frame}
  \frametitle{Thread-Safe}
  Thread safety is about \alert{data}, not \emph{threads}.

  Loosely defined as `no less ``correct'' in multi-threaded setting'.

  Methods of insuring it breaks down into:
  \begin{itemize}
  \item \alert{Don't share} state variables.
  \item Make shared state \alert{immutable}.
  \item Use \alert{synchronisation} to access shared state.
  \end{itemize}

  Object orientation helps here since it can be used to encapsulate
  state. 
\end{frame}

\begin{frame}
  \frametitle{Example}
  \begin{itemize}
  \item Consider program that needs a \emph{long} time to run.
  \item How about adding periodic snapshots to allow restarting computation in
    case of failures.
  \begin{center}
    \includegraphics[height=4cm]{images.jpg}
  \end{center}
  \item Now, we need to synchronise all saved state variables.
  \end{itemize}
\end{frame}

\begin{frame}
  \frametitle{Race Conditions}
  When the \emph{correctness} of a program depends on the timing/interleaving
  of operations (or interference).

  \begin{itemize}
  \item We strive for \alert{atomicity}---at any observable point in time,
    either all or none of the operation has executed.
  \item Beware of \alert{compound actions}
    \begin{itemize}
      \item check-then-act, and
      \item read-modify-write.
    \end{itemize}
  \end{itemize}
  \hfill\includegraphics[height=3cm]{spok2.jpg}
\end{frame}

\begin{frame}
  \frametitle{Reordering}
  The JVM (Java Virtual Machine) is allowed to \alert{reorder} instructions.
  \includegraphics[height=3cm]{spok.jpg}
\end{frame}

\begin{frame}[fragile]
  \frametitle{What could be printed?}
\tiny
\begin{verbatim}
public class PossibleReordering {
    static int x = 0, y = 0;
    static int a = 0, b = 0;

    public static void main(String[] args) throws InterruptedException {
        Thread one = new Thread(new Runnable() {
            public void run() {
                a = 1;
                x = b;
            }
        });
        Thread other = new Thread(new Runnable() {
            public void run() {
                b = 1;
                y = a;
            }
        });
        one.start(); other.start();
        one.join(); other.join();
        System.out.println("( " + x + "," + y + ")");
    }
}
\end{verbatim}
\normalsize
\end{frame}

\begin{frame}
  \frametitle{Happens-Before}
  \begin{itemize}
  \item Program order rule. (Within thread)
  \item Monitor lock rule.
  \item Volatile variable rule.
  \item Thread start rule.
  \item Thread termination rule.
  \item Interruption rule.
  \item Finalizer rule.
  \item Transitivity.
  \end{itemize}
  \hfill\includegraphics[height=3cm]{ghostbusters.jpg}
\end{frame}

\begin{frame}
  \hfill\includegraphics[height=4cm]{spok2.jpg}\hfill\null
\end{frame}

\end{document}

%%% Local Variables: 
%%% mode: latex
%%% TeX-master: t
%%% End: 
