\documentclass[notheorems]{beamer}%[handout]

\mode<presentation>
{
  \useinnertheme{rounded}
  \usecolortheme{rose,seahorse}
  \setbeamercovered{transparent}
  \usefonttheme{serif}
}

%\setbeameroption{show only notes}

%\usepackage[english]{babel}
\usepackage[utf8]{inputenc}
\usepackage{palatino}
\usepackage{tikz}
\usetikzlibrary{arrows,shapes}
\usepackage{fancyvrb}

\theoremstyle{definition}
\newtheorem{theorem}{Theorem}
\newtheorem{lemma}[theorem]{Lemma}
\newtheorem{proposition}[theorem]{Proposition}
\newtheorem{corollary}[theorem]{Corollary}
\newtheorem{definition}[theorem]{Definition}
\newtheorem{example}[theorem]{Example}
\newtheorem{remark}[theorem]{Remark}
\newtheorem{fact}[theorem]{Fact}
\newtheorem{question}[theorem]{Question}
%\theoremstyle{proof}
%\theoremstyle{example}
%\newtheorem{sketch}[theorem]{Sketch}

\DeclareMathOperator{\loglog}{loglog}

\newcommand{\NN}{\ensuremath{\mathbb{N}}}
\newcommand{\QQ}{\ensuremath{\mathbb{Q}}}
\newcommand{\RR}{\ensuremath{\mathbb{R}}}
\newcommand{\ZZ}{\ensuremath{\mathbb{Z}}}
\newcommand{\CC}{\ensuremath{\mathbb{C}}}


\title%[Domain representations of topological algebras]
{Concurrency in Haskell}

\author%[Author, Another]
{Jens Blanck}

\institute% (optional, but mostly needed)
{%
   Swansea University
}

\date%[Short Occasion]
{}

\subject{CS-210 Concurrency}

%\pgfdeclareimage[height=4cm]{turing}{Turing}

\begin{document}

\begin{frame}
  \titlepage
\end{frame}

\begin{frame}
  \frametitle{Concurrency in Haskell}
  \begin{itemize}
  \item Like in Java, Haskell has \alert{threads}.
  \item We can also (as in Java) create Operating system processes.
  \end{itemize}
\end{frame}

\begin{frame}
  \frametitle{Concurrency and Parallelism}
  For the purpose of this discussion we define:
  \begin{itemize}
  \item Parallelism is about making things run faster.
  \item Concurrency is a program-structuring technique that allows multiple
    threads of control to execute simultaneously.
  \end{itemize}
  It is often claimed that parallelism implies concurrency. However, this is
  not necessarily true in a language like Haskell with referential transparency.
\end{frame}

\begin{frame}[fragile]
  \frametitle{The Notion of Threads}
  \begin{itemize}
  \item A thread is something that can run code.
  \item As Java is object oriented we expect a thread to be an \alert{object}.
  \item Indeed, it is. (It is an object that implements the \verb-Runnable-
    interface.)
  \item But Haskell is a \alert{purely} functional language.
  \item Threads would like to interact with IO, so a thread in
    Haskell needs to have an IO type, just like main.
  \end{itemize}
\end{frame}

\begin{frame}[fragile]
  \frametitle{Creating a thread in Haskell}
  \begin{itemize}
  \item Threads in Haskell are created by the \verb-forkIO- function.
  \item \verb+forkIO :: IO () -> IO ThreadId+
  \item The \verb-ThreadId- gives the parent thread a handle to the
    child thread.
  \end{itemize}
  \begin{example}
\begin{verbatim}
main = do
  -- Child thread
  forkIO $ putStrLn "Hello world"
  -- Parent/main thread
  putStrLn "Goodbye cruel world"
\end{verbatim}
  \end{example}
\end{frame}

\begin{frame}[fragile]
  \frametitle{Reminders}
\begin{verbatim}
import Control.Concurrent
import Text.Printf

main :: IO ()
main = loop
  where
    loop = do
      s <- getLine
      if s == "exit"
        then return ()
        else do forkIO $ setReminder s
                loop

setReminder :: String -> IO ()
setReminder s = do
  let t = read s :: Int
  printf "Reminder set for %d s.\n" t
  threadDelay (10^6 * t)
  printf "%d s is up.\n" t
\end{verbatim}
\end{frame}

\begin{frame}[fragile]
  \frametitle{Communication in Haskell}
  \begin{itemize}
  \item Haskell is a purely functional language, which means that it does not
    have \alert{side effects}.
  \item Communicating values correspond to a change of state. Any state change
    is a side effect.
  \item So communication needs to be wrapped into something that can handle
    state changes.
  \item Examples of wrappings relevant for concurrency include \verb-IO- and
    \verb-STM- (Software Transactional Memory).
  \item Both \verb-IO- and \verb-STM- are monads, allowing us to use
    \verb-do--notation.
  \end{itemize}
\end{frame}

\begin{frame}[fragile]
  \frametitle{MVar}
  An \verb-MVar t- is mutable location that is either empty or contains a
  value of type \verb-t-. It has two fundamental operations:
  \begin{itemize}
  \item \verb-putMVar- which fills an \verb-MVar- if it is empty and
    blocks otherwise, and
  \item \verb-takeMVar- which empties an \verb-MVar- if it is full and
    blocks otherwise.
  \end{itemize}

  They can be used in multiple different ways:
  \begin{enumerate}
  \item As synchronized \alert{mutable variables},
  \item As \alert{channels}, with \verb-takeMVar- and \verb-putMVar- as
    receive and send, and
  \item As a \alert{binary semaphore} \verb-MVar ()-, with \verb-takeMVar- and
    \verb-putMVar- as \verb-wait- (\verb-down-) and \verb-signal- (\verb-up-).
  \end{enumerate}
\end{frame}

\begin{frame}[fragile]
  \frametitle{MVar operations}
  \begin{itemize}
  \item \verb=newEmptyMVar :: IO (MVar a)=
  \item \verb=newMVar :: a -> IO (MVar a)=
  \item \verb=takeMVar :: MVar a -> IO a=
  \item \verb=putMVar :: MVar a -> a -> IO ()=
  \end{itemize}
\end{frame}

\begin{frame}
  \frametitle{MVar v Monitors}
  \begin{description}
  \item[Encapsulated data:] Yes. You take the whole chunk of data as one
    unit. Individual pieces cannot be accessed by different threads.
  \item[Access methods:] No. You get the whole chunk, it's your responsibility
    to unpack and repack the data.
  \item[Mutual exclusive access:] Yes. When you take the data out of the MVar
    it is yours until you put it back.
  \item[Condition synchronisation:] Sort of. If you notice that you can't
    proceed, you have to put the original data back in the MVar.
  \end{description}
\end{frame}

\begin{frame}[fragile]
  \frametitle{First MVar Example}
\begin{verbatim}
import Control.Concurrent

main :: IO ()
main = do
  t <- newEmptyMVar
  forkIO $ putMVar t "World"
  putStr "Hello "
  s <- takeMVar t
  putStrLn s
\end{verbatim}
  The MVar acts as \alert{synchronisation} as well as
  \alert{communication}.
\end{frame}

\begin{frame}[fragile]
  \frametitle{Second MVar Example}
\begin{verbatim}
import Control.Concurrent

main :: IO ()
main = do
  t <- newEmptyMVar
  forkIO $ do
    putMVar t "Hello "
    putMVar t "World!"
  s <- takeMVar t
  putStr s
  s <- takeMVar t
  putStrLn s
\end{verbatim}
\end{frame}

\end{document}

%%% Local Variables: 
%%% mode: latex
%%% TeX-master: t
%%% End: 
