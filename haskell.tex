\documentclass[notheorems]{beamer}%[handout]

\mode<presentation>
{
  \useinnertheme{rounded}
  \usecolortheme{rose,seahorse}
  \setbeamercovered{transparent}
  \usefonttheme{serif}
}

%\setbeameroption{show only notes}

%\usepackage[english]{babel}
\usepackage[utf8]{inputenc}
\usepackage{palatino}
\usepackage{tikz}
\usetikzlibrary{arrows,shapes}
\usepackage{fancyvrb}

\theoremstyle{definition}
\newtheorem{theorem}{Theorem}
\newtheorem{lemma}[theorem]{Lemma}
\newtheorem{proposition}[theorem]{Proposition}
\newtheorem{corollary}[theorem]{Corollary}
\newtheorem{definition}[theorem]{Definition}
\newtheorem{example}[theorem]{Example}
\newtheorem{remark}[theorem]{Remark}
\newtheorem{fact}[theorem]{Fact}
\newtheorem{question}[theorem]{Question}
%\theoremstyle{proof}
%\theoremstyle{example}
%\newtheorem{sketch}[theorem]{Sketch}

\DeclareMathOperator{\loglog}{loglog}

\newcommand{\NN}{\ensuremath{\mathbb{N}}}
\newcommand{\QQ}{\ensuremath{\mathbb{Q}}}
\newcommand{\RR}{\ensuremath{\mathbb{R}}}
\newcommand{\ZZ}{\ensuremath{\mathbb{Z}}}
\newcommand{\CC}{\ensuremath{\mathbb{C}}}


\title%[Domain representations of topological algebras]
{Concurrency in Haskell}

\author%[Author, Another]
{Jens Blanck}

\institute% (optional, but mostly needed)
{%
   Swansea University
}

\date%[Short Occasion]
{}

\subject{CS-210 Concurrency}

%\pgfdeclareimage[height=4cm]{turing}{Turing}

\begin{document}

\begin{frame}
  \titlepage
\end{frame}

\begin{frame}
  \frametitle{Concurrency in Haskell}
  \begin{itemize}
  \item Like in Java, Haskell has \alert{threads}.
  \item We can also (as in Java) create Operating system processes.
  \end{itemize}
\end{frame}

\begin{frame}
  \frametitle{Concurrency and Parallelism}
  For the purpose of this discussion we define:
  \begin{itemize}
  \item Parallelism is about making things run faster.
  \item Concurrency is a program-structuring technique that allows multiple
    threads of control to execute simultaneously.
  \end{itemize}
  It is often claimed that parallelism implies concurrency. However, this is
  not necessarily true in a language like Haskell with referential transparency.
\end{frame}

\begin{frame}[fragile]
  \frametitle{The Notion of Threads}
  \begin{itemize}
  \item A thread is something that can run code.
  \item As Java is object oriented we expect a thread to be an \alert{object}.
  \item Indeed, it is. (It is an object that implements the \verb-Runnable-
    interface.)
  \item But Haskell is a \alert{purely} functional language.
  \item Threads would like to interact with IO, so a thread in
    Haskell needs to have an IO type, just like main.
  \end{itemize}
\end{frame}

\begin{frame}[fragile]
  \frametitle{Creating a thread in Haskell}
  \begin{itemize}
  \item Threads in Haskell are created by the \verb-forkIO- function.
  \item \verb+forkIO :: IO () -> IO ThreadId+
  \item The \verb-ThreadId- gives the parent thread a handle to the
    child thread.
  \end{itemize}
  \begin{example}
\begin{verbatim}
main = do
  -- Child thread
  forkIO $ putStrLn "Hello world"
  -- Parent/main thread
  putStrLn "Goodbye cruel world"
\end{verbatim}
  \end{example}
\end{frame}

\end{document}

%%% Local Variables: 
%%% mode: latex
%%% TeX-master: t
%%% End: 
